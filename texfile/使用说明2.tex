% 1. 用户提问 Block(浅橙色,结尾附具体使用目的和环节说明)
\begin{block}[colback=orange!10, colframe=orange!60, title={\textcolor{black}{用户提问}}]
如何计算两个虚数的和?
\vspace{0.5em}
\begin{flushright}
    \scriptsize
    \begin{flushleft}
        \textbf{具体使用目的和环节:} 本次调用 AI 工具用于解答数学基础问题,环节为“算法设计与实现”,目的是快速获得标准解法并规范表达步骤。
    \end{flushleft}
\end{flushright}
\end{block}

% 2. AI 回答 Block 环境定义(自动带图标、标题、模型信息)
% 用法:\begin{geminiblock} ... \end{geminiblock}

% 用法示例
\begin{GeminiBlock}
以下是计算两数之和的简单算法:
\begin{enumerate}
    \item 输入两个虚数 $x$ 和 $y$;
    \item 计算 $s = x + y$;
    \item 输出结果 $s$。
\end{enumerate}
\end{GeminiBlock}

% 2b. 用户追问 Block(再次交互,展示追问以获取形式化结果与示例)
\begin{block}[colback=orange!10, colframe=orange!60, title={\textcolor{black}{用户追问}}]
如果把两个虚数具体写成代数形式 $(a+bi)$ 与 $(c+di)$,其和应如何规范表示?能否再给一个带具体数字的示例来说明?
\vspace{0.5em}
\begin{flushright}
    \scriptsize
    \begin{flushleft}
        	extbf{追问目的:} 进一步明确复数加法的通式表达与数值示例,便于在论文中直接引用;仍处于“算法设计与实现”环节。
    \end{flushleft}
\end{flushright}
\end{block}

% 2c. AI 对追问的回答
\begin{GeminiBlock}
复数(虚数是其特殊情形)加法的通式:
\[
(a+bi) + (c+di) = (a+c) + (b+d)i.
\]
其中实部相加,虚部系数相加,保持 $i$ 不变。

数值示例:
\[
(2+3i) + (1-5i) = (2+1) + (3-5)i = 3 - 2i.
\]
说明:该运算不需要额外条件,时间复杂度 $O(1)$;在实现层面可分别存储实部、虚部再组合输出。
\end{GeminiBlock}

% 3. 采纳和人工修改情况
\begin{block}[colback=green!10, colframe=green!50, title={\textcolor{black}{采纳和人工修改情况}}]
\begin{itemize}
    \item \textbf{交互轮次:} 2(初次提问 + 1 次追问)。
    \item \textbf{采纳情况:} 初次回答与追问回答均被完全采纳。
    \item \textbf{人工修改情况:} 无需人工修改,AI 给出的通式与示例已直接满足报告中“算法说明”小节的引用需求。
\end{itemize}
\end{block}
